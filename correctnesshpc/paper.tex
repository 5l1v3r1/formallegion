\documentclass[sigconf]{acmart}

\usepackage{booktabs} % For formal tables
\usepackage{listings} % For formal tables


% Copyright
%\setcopyright{none}
%\setcopyright{acmcopyright}
%\setcopyright{acmlicensed}
\setcopyright{rightsretained}
%\setcopyright{usgov}
%\setcopyright{usgovmixed}
%\setcopyright{cagov}
%\setcopyright{cagovmixed}


% DOI
\acmDOI{10.475/123_4}

% ISBN
\acmISBN{123-4567-24-567/08/06}

%Conference
\acmConference[Correctness'17]{Correctness 2017: First International Workshop on
  Software Correctness for HPC Applications}{November 2017}{Denver Colorado, USA} 
\acmYear{1997}
\copyrightyear{2016}

\acmPrice{15.00}

\author{George Stelle}
\authornote{where does this go? what's an orcid?}
\orcid{1234-5678-9012}
\affiliation{%
  \institution{Los Alamos National Laboratories}
}
\email{gstelle@lanl.gov}

\author{Noah Evans}
\authornote{where?}
\affiliation{%
  \institution{Sandia National Laboratories}
  \streetaddress{P.O. Box 1212}
  \city{Dublin} 
  \state{Ohio} 
  \postcode{43017-6221}
}
\email{nevans@sandia.gov}
\title{Towards Correct Legion}
\begin{document}
\maketitle{}<++>
\begin{abstract}
Legion is a programming model that uses logical regions of memory to reason about 
locality and concurrency. We formalize the Legion semantics in a mechanized
proof assistant, Coq, and show how this enables one to reason formally and
prove theorems about the semantics. We also discuss how to extend this work to
prove that serial semantics are preserved and how given enough time, one could
extend the effort to prove an implementation is correct.
\end{abstract}

\section{Intro}
High performance computing, as the name implies, has long focused on
performance, and for good reason. Expensive machines are purchased to run
increasingly complex simulations, and every optimization can save huge amounts
of money, both due to the initial cost of the machine and the cost of power to
run it. 

That said, what good is a fast computation if we aren't relatively certain that
the answers it returns are correct? Presumably, these computations inform
important decisions about topics ranging from climate to nuclear safety. We
should be as confident as possible that we aren't basing these decisions on
unexpected semantics.

\section{Background}

Legion is a programming model for high performance computing that enables
reasoning about locality and with the attractive property that it is guaranteed
to preserve serial semantics. It achieves this by dividing memory into logical
regions.  This allows programmers to avoid the
burden 

Improvements to proof assistants over recent decades have made them a powerful
tool for implementing and reasoning about code. From provably correct compilers
to provably correct operating systems, proof assistants enable extremely high
certainty in correctness even for large scale software projects. 

Coq is a dependently typed proof assistant. At its core is the Calculus of
Inductive Constructions, a small dependently typed programming language and
logic, capable of powerful abstractions for formal proofs. In the last 10 years 
great strides have been made in using it to both build systems and prove useful
properties about them \cite{compcert, certicoq, chlipala...}. 

Specifications come in a wide range of formats. Many, like the OpenMP
specification, are very informal, primarily consisting of english. Others, like
Legion's previously, are written by hand in the using standard notation for
logical relations, etc. For this paper, when we refer to a \emph{formal}
specification, we are referring to a specification defined in a formal logic,
like Coq. 

While semantics defined in papers are a huge improvement over natural language
specifications, there is still room for improvement. \cite{formalspec vs paper} has 
shown that even mathematical specifications and proofs in computer science
papers almost always have errors that would be caught by a formal proof. This is not to say,
of course, that formal specifications and proofs are infallible. On the contrary, it 
is quite easy to get a specification wrong. That said, often, if a there is a mistake in a spec
, the process of proving it formally will expose that mistake, allowing for its correction. 

\section{A Formal Specification of Legion}

"A program without a specification cannot be incorrect, it can only be
surprising." TODO: Find quote source in recent Appel paper. One of the most
important and challenging aspects of proving correctness is defining a formal
specification of what a program *should* do. Indeed, there is currently a large
NSF effort dedicated to improving our ability to do just this \cite{deepspec}. 

One thing that makes Legion unique among high performance runtime systems is
that it provides a specification of its operational semantics \cite{oopsla13}.
We take this specification of and formalize it in Coq, enabling us to prove
lemmas and theorems about it.

\section{Discussion}

One common theme in formalizing systems is the discovery of holes in the
corresponding informal specification. For example, when formalizing the C
language for the CompCert project, Leroy et. al. found many cases where the
specification was incomplete. Note that this is seperate notion from undefined
behavior. Undefined behavior in a specification is well defined. A little bit
like known unknowns vs. unknown unknowns, gaps in a specification  

In formalizing Legion, we discovered that while there is an impressive theory
ensuring that the side effects of function calls running concurrently preserve
serial semantics, there is no theory for what function calls can run
concurrently based on value dependencies. For example, consider the infamous
Fibonacci example, which naively computes the \texttt{n}'th Fibonacci number: 

\begin{lstlisting}
fib(n) = if n < 2 then 1 else fib(n-1) + fib(n-2)
\end{lstlisting}
 
Of course 

\end{document}
