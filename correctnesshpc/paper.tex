\documentclass[sigplan]{acmart}

\usepackage{booktabs} % For formal tables
\usepackage{listings} % For formal tables

% Copyright
%\setcopyright{none}
%\setcopyright{acmcopyright}
%\setcopyright{acmlicensed}
\settopmatter{printacmref=false}
\setcopyright{rightsretained}
%\setcopyright{usgov}
%\setcopyright{usgovmixed}
%\setcopyright{cagov}
%\setcopyright{cagovmixed}


% DOI
\acmDOI{10.475/123_4}

% ISBN
\acmISBN{123-4567-24-567/08/06}

%Conference
\acmConference[Correctness'17]{Correctness 2017: First International Workshop on
  Software Correctness for HPC Applications}{November 2017}{Denver Colorado, USA} 
\acmYear{2017}
\copyrightyear{2017}

%\acmPrice{15.00}

\begin{document}

\title{Towards Correct Legion}
\author{George Stelle}
\affiliation{\institution{Los Alamos National Laboratories}}
\email{stelleg@lanl.gov}

\author{Noah Evans}
\affiliation{\institution{Sandia National Laboratories}}
\email{nevans@sandia.gov}

\begin{abstract}
Legion is a programming model that uses logical regions of memory to reason about 
locality and concurrency. We formalize the Legion semantics in a mechanized
proof assistant, Coq, and show how this enables one to reason formally and
prove theorems about the semantics. We also discuss how to extend this work to
prove that serial semantics are preserved and how given enough time, one could
extend the effort to prove an implementation is correct.
\end{abstract}

\maketitle

\section{Intro}
High performance computing, as the name implies, has long focused on
performance. Expensive machines are purchased to run increasingly complex
simulations, and every optimization can save huge amounts of money, both due to
the initial cost of the machine and the cost of power to run it. 

That said, what good is a fast computation if we aren't certain that the
values it computes are correct? Presumably, these computations inform important
decisions about topics ranging from climate to nuclear safety. We should be as
confident as possible that we aren't basing these decisions on unexpected
semantics.

Recent work has shown that modular, composable formal specifications enable
formal correctness proofs for large software systems that were previously
unattainable. By integrating a formal specification tightly with an
implementation, one can achieve the highest levels of confidence in code
currently possible \cite{?}. For example, the Compcert project implements an
optimizing c compiler and proves it correct. The success of Compcert in
removing compiler bugs has been well documented. We will discuss later how 
Compcert could be used in building verifiable HPC systems.

In HPC, programming models have become an increasingly important way to ensure
that code stays maintanable, performant, and portable. We argue that these
programming models provide a good target for formalization: any correctness
arguments can be leveraged by every program using the model. It is exactly this 
kind of modularization of proofs that has been so successful in proving programs 
correct \cite{?}.

For this paper, we take a recent programming model, Legion, and formalize its
semantics and type system. We then prove a few supporting lemmas as examples of
the kinds of things one can prove in an HPC settings. Finally, we discuss the
some of the possible paths towards full, formally correct implementation. This
is no small task, as we discuss later, and the ways forward are numerous and
lengthy. But if we believe that HPC informs important decisions, we must strive
for this pinnacle of software correctness!

\section{Background}

Legion is a programming model for high performance computing that enables
reasoning about locality and with the attractive property that it is guaranteed
to preserve serial semantics. It achieves this by dividing memory into logical
regions.  This allows programmers to avoid the
burden 

Improvements to proof assistants over recent decades have made them a powerful
tool for implementing and reasoning about code. From provably correct compilers
to provably correct operating systems, proof assistants enable extremely high
certainty in correctness even for large scale software projects. 

Coq is a dependently typed proof assistant. At its core is the Calculus of
Inductive Constructions, a small dependently typed programming language and
logic, capable of powerful abstractions for formal proofs. In the last 10 years 
great strides have been made in using it to both build systems and prove useful
properties about them \cite{compcert, certicoq, chlipala...}. 

Specifications come in a wide range of formats. Many, like the OpenMP
specification, are very informal, primarily consisting of english. Others, like
Legion's previously, are written by hand in the using standard notation for
logical relations, etc. For this paper, when we refer to a \emph{formal}
specification, we are referring to a specification defined in a formal logic,
like Coq. 

While semantics defined in papers are a huge improvement over natural language
specifications, there is still room for improvement. \cite{formalspec vs paper} has 
shown that even hand-written mathematical specifications and proofs in computer
science papers almost always have errors that can be caught by a formal proof.
This is not to say, of course, that formal specifications and proofs are
infallible.  On the contrary, it is quite easy to get a specification wrong.
That said, often, if a there is a mistake in a spec , the process of proving it
formally will expose that mistake, allowing for its correction. 

\section{A Formal Specification of Legion}

"A program without a specification cannot be incorrect, it can only be
surprising." \cite{appel}. One of the most important and challenging aspects of
proving correctness is defining a formal specification of what a program
*should* do. Indeed, there is currently a large NSF effort dedicated to
improving our ability to do just this \cite{deepspec}. The choice of
specification has far-reaching consequences in how easy it is to later prove
correctness properties. For example, if a specification is chosen poorly, 

One thing that makes Legion unique among high performance runtime systems is
that it provides a specification of its operational semantics \cite{oopsla13}.
We take this specification of and formalize it in Coq, enabling us to prove
lemmas and theorems about it. See Figure~\ref{?} for the Coq implementation of the
syntax and semantics.

The Legion semantics are defined as a small step operational semantics, which
result in non-deterministic orderings of memory reads and writes. By defining a
notion of \texttt{valid\_interleaves}, we can reason formally about what evaluations 
can happen concurrently. This enables Treichler et al. to prove that if the runtime
makes scheduling decisions based on region permissions in a particular way, serial 
semantics are preserved with respect to the resulting heap state and return
value. This is a desirable property: it means that any code using Legion can
reason, either formally or informally, about its code with respect to a simple
serial semantics, and have those semantics be preserved for parallel execution.
This is in contrast to programming models like Cilk, which claim a weaker
property: a serial execution is one of many valid executions. 

\section{Discussion}

One common theme in formalizing systems is the discovery of holes in the
corresponding informal specification. For example, when formalizing the C
language for the CompCert project, Leroy et. al. found many cases where the
specification was incomplete. Note that this is seperate notion from undefined
behavior. Undefined behavior in a specification is well defined. A little bit
like known unknowns vs. unknown unknowns, gaps in a specification can make it
impossible to prove desired properties of a system.  

In formalizing Legion, we discovered that while there is an impressive theory
ensuring that the side effects of function calls running concurrently preserve
serial semantics, there is no theory for what function calls can run
concurrently based on value dependencies. For example, consider the infamous
Fibonacci example, which naively computes the \texttt{n}'th Fibonacci number: 

\begin{lstlisting}
fib(n) = if n < 2 then 1 else fib(n-1) + fib(n-2)
\end{lstlisting}

\subsection{Performance}

Returning to the "P" in "HPC", an important area of research is formal
reasoning about performance. One of the challenges for Legion is that the
scheduler has a non-trivial amount of work to do. Generally at least one core
per node is dedicated entirely to runtime overheads. This is one area where
Cilk, though fairly informally, shines. They prove a number of bounds on work
that the scheduler has to do.  

Ideally, we'd like to make similar claims about Legion runtime overheads.
Currently, due to the complexity of the runtime implementation, this is
infeasible. Like some of the desired correctness properties, a significant
simplification of the runtime, and perhaps the semantics, would likely be
required to achieve such a goal.

\section{Conclusion}

We hope to have convinced the reader of the importance and viability of proving
software systems in HPC correct. While we've only taken small steps on the path
to provably correct Legion, our hope is that it will inspire support for such 
ventures by providing a convincing story on how to move forward. There is ever-
increasing evidence that this class of deep specifications and certified software
is an incredibly powerful tool for proving desirable properties about our large 
software systems. We'd like to see more focus on correctness in HPC, and we think
deep specifications and certified proofs are the way forward.

 \bibliographystyle{ACM-Reference-Format}
 \bibliography{annotated}
\end{document}
